\documentclass[10pt,a4paper]{article}
\usepackage[utf8]{inputenc}
\usepackage{amsmath}
\usepackage{amsfonts}
\usepackage{amssymb}
\usepackage{ulem}
\usepackage{graphicx}
\author{GIMENEZ Florian}
\title{Compte Rendu TP4 Mise à Niveau Programmation}

\begin{document}
\normalem

\section{Traveaux Pratique 4}

\subsection{Exercice 1}
\paragraph{}
    Ici la variable \emph{p} est un pointeur sur le tableau. Au premier printf, p nous donne la valeur de la première case.
    Lorsqu'on incrémente P on va pouvoir afficher les cases suivantes du tableaux.





\end{document}